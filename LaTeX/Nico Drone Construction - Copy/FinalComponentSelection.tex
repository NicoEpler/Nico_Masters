\chapter{Final Component Selection}

\section{Our current team and their focus}
	
	A very brief overview of our project team is shown below. This will assist you in speaking to the right team member should you require some assistance. Otherwise speak to the PI (principal investigator, Callen Fisher) for assistance.
	
	\begin{table}[H]
		\caption{Our current team and their focus}
		\label{tab:ourTeam}
		\begin{tabular}{|l|l|l|l|l|}
			\hline
			Name          & Degree & Focus      & email address     & Contact no.\\
			\hline
			Callen Fisher & PI    & PI          & cfisher@sun.ac.za  & 0836626669\\
			JJ Barnard    & MEng  & Rover       & 23782994@sun.ac.za & \\
			N Epler       & MEng  & Drone       & 23910712@sun.ac.za & 0826537180\\
			Z Wu		  & MEng  & Navigation  & 					  & \\
			AD Sendzul    & MEng  & Sensor Pack & 23557702@sun.ac.za & 0823412651\\
			              &        &                 &                    & \\
			              &        &                 &                    & \\
			              &        &                 &                    & \\
			              &        &                 &                    & \\
			              &        &  	             &                    & \\ \hline 
		\end{tabular}
	\end{table}

\section{NDA}
	To work on this project, you will be required to sign an NDA with the university. This means you cannot share or discuss your code, work or any knowledge you gain, with someone that has not signed the NDA. Japie Engelbrecht and Willem Jordaan are two other academics that have signed this NDA as well, so you can discuss technical details with them.
	
	This does not mean you cannot talk about your research. Just please err on the side of caution when doing so. Please do not share any code or technical information with anyone and do not put it on the internet for others to use. 
	
	You can discuss the high level things about your research, but if people start probing into what you are doing, please do not provide any technical information. 

\section{GIT}
	This document is on GIT, please become familiar with how GIT works. Additionally, I will make a GIT repo for each of you, all of your code and work needs to be on this GIT. Therefore, I will have access to it at all times, as well as when you leave. 
	
	GIT is an amazing tool that keeps your work backed up in case of failures. Please make full use of it and commit frequently. If you have never used GIT before, please watch this video for an intro and some tips:
	
	People rate this video: \url{https://www.youtube.com/watch?v=tRZGeaHPoaw}, personally I read the ProGit manual and it helped me a bunch so I'd recommend that (freely available online).

\section{This document}
	This document is aimed at being a communal resource for current and new students working on this project. As there is lots of overlap between the different projects, I want to minimize the amount of duplicated research, so all research (links, algorithms etc) will need to be documented in this document. Additionally, this will ensure that we all take the same approach when it comes to what sensors we are using etc. Please keep this up to date at all times. It will also aid me to determine how the project is progressing and if we are sticking to our time lines. 
	
	Please note this document will never be published, so you do not have to keep it formal or professional, but it must remain functional and beneficial for the team. 

\section{Weekly meetings}
	Please make full use of these meetings. This is a time to discuss the problems you have been facing and brain storm new approaches or to discuss bugs you have experienced. Hopefully the team will be able to find solutions to these problems. Additionally, if you have found useful articles that others should read, please share them in these meetings. 

	Communication between members of the team and PI is crucial. We have to work as a team, otherwise some members will be re-doing what others have just done, wasting time. That being said, every member of the team needs to pull their weight. 
	
\section{Working from home}
	On occasion, you may work from home if you find it more productive. All meetings will be in person, unless prior arrangements with a valid reason. Please note, I expect everyone to be in at least 4 times a week. This will aid in working as a team as well as me keeping track of everyone's progress. I am frequently in the lab and will be available to answer questions and provide guidance where needed. 

\section{Progress reports}
	I have to provide quarterly progress reports on this project. This document will assist me in doing so. It is crucial that you frequently attend weekly meetings and provide me with a progress update. If you are stuck, ask for assistance. As can be seen, there are some very tight deadlines, with a working prototype that needs to be delivered in 3 years time (end of 2026).  
	
\section{Publications}
	Enaex has the right to review any and all publications (conference papers, journals, dissertation) that get submitted. They may request changes be made as well. 
	
	Therefore, when writing for a conference, we have to factor this delay into the submission date as well. Additionally, your work and writing has to be of the highest standard as my reputation is on the line. I will be very critical when proof reading your work, and nothing can be submitted without my approval and thorough proof reading. 
	
\section{References}
	Please add a bibliography section and keep it up to date. Add references to all content you add to this document. You can then use this document, along with everyones input to write up your thesis. This will also assist in quickly writing up publications. 